The design of the LogDAG will be tested through
a simulated environment
generating the LogDAG.
This LogDAG will then be visualized as a graph.
Our setup is described in this section.
The code has been published on \texttt{github.com/svlentink/LogDAG}.

\subsubsection{Nodes}
We use five participating nodes,
which are docker containers simulating a web server,
generating access logs.
A sixth docker container runs a script that queries
the web servers, causing them to generate access logs.
All six containers are joined on a private network,
configured in a \texttt{docker-compose.yml} file.

The five nodes have a web server,
cron functionality,
and the features needed to synchronize the LogDAG,
as show in figure \ref{fig:host}.

\subsubsection{Visualization}
\label{method_visualization}

The visualization of the LogDAG is done by generating a graph
based on the validations of the blocks.
The GUI is a website which loads the LogDAG via an AJAX request.
The data from the LogDAG is stored in a JSON file,
which is visualized using d3js,
a library used for visualization.

The GUI is a seventh container launched by the
\texttt{docker-compose.yml},
showing the increasing LogDAG in real-time.
This enables one to see the graph evolving over time
and create a deeper understanding.
It allows one to try different variables for $v$.
