In this section we describe the design of the LogDAG,
discussing the various components by giving an example design.
The design of the LogDAG is inspired by the IOTA Tangle\cite{popov2016tangle}.
The components discussed should be
seen as example solutions in a modular system,
%implemented as a modular solution,
enabling different protocols or algorithms to be chosen,
creating a layered solution.

In this section we will use $n$ for the amount of nodes participating in the LogDAG.
The rotate time interval of the log data is denoted by $r$
and the amount of blocks to validate by $v$.

\subsection{Blocks}
We begin with describing the units that contain the actual data,
which we call blocks.
These blocks contain log data of time interval $r$,
which are created by rotating log files
as shown in figure \ref{fig:formation}.

\begin{figure}[H]
\centering \includegraphics[width=.5\columnwidth]{block-formation.png}
\caption{ \label{fig:formation}  
Log block formation
}
\end{figure}

Log files are merged into a compressed block,
which is signed by the key of the host that creates the block
and is encrypted with a public key.
The private key corresponding to this public key should not be stored on any participating node,
we call this key the \textit{auditor key pair}.

\subsection{Block metadata}\label{metadata}

To identify a block we create a block metadata object.
This metadata object contains a minimum of the following attributes:

\begin{lstlisting}
{
  blockid : 'hash_c',
  hostname : 'node01',
  time : 12345,
  validates : [ 'hash_a', 'hash_b' ]
}
\end{lstlisting}

The time is used for the \ttl,
the retention period of the logs.
%The blockid is dependent on the blocks it validates,
%of which the quantity of validations may vary.
The blockid is created using the hash of the $v$ blocks this block validates,
thereby creating a dependency on the previous block(s).
This block creation process is shown in figure \ref{fig:hash}.
This dependency results in a DAG (directed acyclic graph).

\begin{figure}[H]
\centering \includegraphics[width=.4\columnwidth]{blockid.png}
\caption{ \label{fig:hash}  
Creation of blockid
}
\end{figure}

When creating a new blockid,
the linked block(s) are also retrieved and stored on the node which creates the blockid.

\subsection{LogDAG}

The array of block metadata,
which we call the LogDAG,
contains the metadata of all block.
\begin{figure}[H]
\centering \includegraphics[width=1\columnwidth]{host.png}
\caption{ \label{fig:host}  
Participating node
}
\end{figure}
Nodes provide an interface to other nodes
to retrieve blocks it has stored,
which include the blocks it has validated,
as shown in figure \ref{fig:host}.
When new block metadata has been formed by a node,
it is broadcast to all other nodes.
Each node stores this metadata to its copy of the LogDAG.

\subsection{Process}

The process could look as follows:
\begin{lstlisting}
/etc/cron.d/createblock
|
|- rotate_log_files
|- block_formation
|  |- create_compressed_block
|  |- sign_compressed_block
|  '- encrypt_compressed_block
|- create_block_metadata
|  |- decide_which_to_validate
|  |- retrieve_blocks_to_validate
|  |- validate_blocks
|  '- generate_blockid
'- broadcast_block_metadata / add_to_LogDAG
\end{lstlisting}
in which the first line is triggered by the cron daemon.

\subsection{Configuration}

The participating nodes have a shared configuration containing:
\begin{itemize}
\item List of participating nodes
\item \ttl\ for log retention
\item Shared auditor public-key for encrypting the blocks
\item The amount of blocks to validate per block, denoted as $v$
\item List of default log files to include when available (e.g. \texttt{/var/log/auth.log})
\end{itemize}

The nodes itself have additional configuration options,
such as specific logs to include in the block formation.

